% !TeX document-id = {fb8a2ef5-cdaf-49da-b79d-0a8152e677cd}
% !TeX TS-program = XeLaTeX
\documentclass[a4paper,12pt]{report}
\raggedbottom
% polyglossia should go first!
\usepackage{polyglossia} % multi-language support

\setdefaultlanguage{russian}
\setmainfont{CMU Serif}
\setsansfont{CMU Sans Serif}
\setmonofont{CMU Typewriter Text}


\setmainlanguage{russian}
\setotherlanguage{english}

\DeclareSymbolFont{letters}{\encodingdefault}{\rmdefault}{m}{it}
\usepackage{amsmath} % math symbols, new environments and stuff
\usepackage{unicode-math} % for changing math font and unicode symbols
\setmathfont{XITS Math}
\allowdisplaybreaks[2]

\usepackage[style=english]{csquotes} % fancy quoting
\usepackage{microtype} % for better font rendering
\usepackage[backend=bibtex, sorting=none]{biblatex} % for bibliography, TODO: replace with biber
\usepackage{hyperref} % for refs and URLs
\usepackage{graphicx} % for images (and title page)
\usepackage{geometry} % for margins in title page
\usepackage{tabu} % for tabulars (and title page)
\usepackage{placeins} % for float barriers
\usepackage{titlesec} % for section break hooks
%\usepackage[justification=centering]{caption} % for forced captions centering
\usepackage{subcaption} % for subfloats
%\usepackage{rotating} % for rotated labels in tables
%\usepackage{tikz} % for TiKZ
%\usepackage{dot2texi} % for inline dot graphs
\usepackage{listings} % for listings 
\usepackage{upquote} % for good-looking quotes in source code (used for custom languages)
\usepackage{multirow} % for multirow cells in tabulars
%\usepackage{afterpage} % for nice landspace floats
%\usepackage{pdflscape} % for landspace orientation
\usepackage{xcolor} % colors!
\usepackage{enumitem} % for unboxed description labels (long ones)
\usepackage{lastpage}
\usepackage{subcaption}
\usepackage{tikz}
\usetikzlibrary{shapes,arrows,patterns}
\usepackage{array}
\newcolumntype{L}[1]{>{\raggedright\let\newline\\\arraybackslash\hspace{0pt}}m{#1}}
\newcolumntype{C}[1]{>{\centering\let\newline\\\arraybackslash\hspace{0pt}}m{#1}}
\newcolumntype{R}[1]{>{\raggedleft\let\newline\\\arraybackslash\hspace{0pt}}m{#1}}

\lstloadlanguages{C,haskell,bash,java}

\usepackage{textcomp}

\lstset{
%    inputencoding=utf8,
%    backgroundcolor=\color{white},
    tabsize=4,
    rulecolor=,
    upquote=true,
%    aboveskip={1.5\baselineskip},
    columns=fixed,
    showstringspaces=false,
    extendedchars=true,
    breaklines=true,
    prebreak = \raisebox{0ex}[0ex][0ex]{\ensuremath{\hookleftarrow}},
    frame=single,
    showtabs=false,
    showspaces=false,
    showstringspaces=false,
    basicstyle=\scriptsize\ttfamily,
    identifierstyle=\ttfamily,
    keywordstyle=\ttfamily\color[rgb]{0,0,1},
    commentstyle=\ttfamily\color[rgb]{0.133,0.545,0.133},
    stringstyle=\ttfamily\color[rgb]{0.627,0.126,0.941},
}

\renewcommand\lstlistingname{Листинг}

\defaultfontfeatures{Mapping=tex-text} % for converting "--" and "---"

\MakeOuterQuote{"} % enable auto-quotation

\newcommand{\varI}[1]{{\operatorname{\mathit{#1}}}}

% new page and barrier after section, also phantom section after clearpage for
% hyperref to get right page.
% clearpage also outputs all active floats:
\newcommand{\sectionbreak}{\clearpage\phantomsection}
\newcommand{\subsectionbreak}{\FloatBarrier}
\newcommand\numberthis{\addtocounter{equation}{1}\tag{\theequation}}
\renewcommand{\thesection}{\arabic{section}} % no chapters
\numberwithin{equation}{section}
%\usetikzlibrary{shapes,arrows,trees}

\usepackage{expl3}
\ExplSyntaxOn
\cs_new_eq:NN \Repeat \prg_replicate:nn
\ExplSyntaxOff


\DeclareMathOperator{\rank}{rank}
\makeatletter

\lstdefinelanguage{llvm}{
  morecomment = [l]{;},
  morestring=[b]", 
  sensitive = true,
  classoffset=0,
  morekeywords={
    define, declare, global, constant,
    internal, external, private,
    linkonce, linkonce_odr, weak, weak_odr, appending,
    common, extern_weak,
    thread_local, dllimport, dllexport,
    hidden, protected, default,
    except, deplibs,
    volatile, fastcc, coldcc, cc, ccc,
    x86_stdcallcc, x86_fastcallcc,
    ptx_kernel, ptx_device,
    signext, zeroext, inreg, sret, nounwind, noreturn,
    nocapture, byval, nest, readnone, readonly, noalias, uwtable,
    inlinehint, noinline, alwaysinline, optsize, ssp, sspreq,
    noredzone, noimplicitfloat, naked, alignstack,
    module, asm, align, tail, to,
    addrspace, section, alias, sideeffect, c, gc,
    target, datalayout, triple,
    blockaddress
  },
  classoffset=1, keywordstyle=\color{purple},
  morekeywords={
    fadd, sub, fsub, mul, fmul,
    sdiv, udiv, fdiv, srem, urem, frem,
    and, or, xor,
    icmp, fcmp,
    eq, ne, ugt, uge, ult, ule, sgt, sge, slt, sle,
    oeq, ogt, oge, olt, ole, one, ord, ueq, ugt, uge,
    ult, ule, une, uno,
    nuw, nsw, exact, inbounds,
    phi, call, select, shl, lshr, ashr, va_arg,
    trunc, zext, sext,
    fptrunc, fpext, fptoui, fptosi, uitofp, sitofp,
    ptrtoint, inttoptr, bitcast,
    ret, br, indirectbr, switch, invoke, unwind, unreachable,
    malloc, alloca, free, load, store, getelementptr,
    extractelement, insertelement, shufflevector,
    extractvalue, insertvalue,
  },
  alsoletter={\%},
  keywordsprefix={\%},
}

\newenvironment{sqcases}{%
  \matrix@check\sqcases\env@sqcases
}{%
  \endarray\right.%
}
\def\env@sqcases{%
  \let\@ifnextchar\new@ifnextchar
  \left\lbrack
  \def\arraystretch{1.2}%
  \array{@{}l@{\quad}l@{}}%
}
\makeatother
\setcounter{tocdepth}{3}

%for abstract
%page count \pageref{LastPage}
\usepackage{lastpage}  
%\totalfigures \totaltables
\usepackage[figure,table,lstlisting,xspace]{totalcount}

\date{\today}

\usepackage{etoolbox}
\newcounter{totreferences}
\pretocmd{\bibitem}{\addtocounter{totreferences}{1}}{}{}
\makeatletter
     \AtEndDocument{%
       \immediate\write\@mainaux{%
         \string\gdef\string\totref{\number\value{totreferences}}%
       }%
     }
\let\thetitle\@title
\let\theauthor\@author
\let\thedate\@date
\makeatother

\bibliography{reference_list} 
\begin{document} 

\newgeometry{margin=1cm}
\begin{titlepage}
  \begin{center}
    \emph{Федеральное государственное бюджетное образовательное учреждение высшего
      профессионального образования}
    \begin{tabu} to \linewidth {lX[1,c,m]}
      \hline \\
      \includegraphics[width=0.15\linewidth]{img/crest} &
      \large\emph{``Московский государственный технический университет имени Н.Э.
        Баумана'' (МГТУ им. Н.Э. Баумана)} \\
    \end{tabu}
  \end{center}
  \begin{tabu}{ll}
    \large\textsc{Факультет:} & Информатика и системы управления \\
    \large\textsc{Кафедра:} & Программное обеспечение ЭВМ и информационные технологии \\
  \end{tabu}
  \vspace{1.0cm}
  \begin{center}
    \huge{Курсовая работа} \\
    \vspace{0.5cm}
    \small{по теме} \\
    \vspace{0.5cm}
    \Large{``Бекенд компилятора языка C--''}
  \end{center}
  \vfill
  \begin{tabu} to \linewidth {Xr}
    Студент: Гребенюк Александр, ИУ7-29& \\
  \end{tabu}
  \vspace{0.2cm}
  \begin{center}
    Москва \the\year
  \end{center}
\end{titlepage}
\restoregeometry

%skip for title page
\setcounter{page}{2}

\section*{Реферат}
Отчёт \pageref{LastPage}с., \totaltables табл., \totallstlistings листингов, 4 источника, 0 приложений.

\noindent \textbf{Компиляторы, LLVM, JIT, Haskell.} 

Предмет работы составляет разработка бекенда компилятора упрощённой версии языка C-- с целью изучения его механизмов работы.

Цель работы - создание программного обеспечения, осуществляющего трансляцию абстрактное синтсаксического дерева в код LLVM IR с последующей JIT-компиляцией. 

Было разработано ПО, осуществляющее трансляцию абстрактное синтсаксического дерева в промежуточное представление кода виртуальной машины LLVM -- LLVM IR и
JIT-компиляцию полученного кода. Был изучен механизм линковки сторонних библиотек для обеспечения работы инструкций ввода-вывода языка C--.

В результате было разработан бекенд компилятора C-- и приведены примеры трансляции АСТ в LLVM IR.

\tableofcontents

\section{Введение}
В настоящей работе в рамках курса ``конструирование компиляторов'' реализуется бекенд компилятора упрощённой версии языка C--.
В работе приводится описание трансляции токенов абстрактного синтсаксического дерева в LLVM IR и примеры обнаруживаемых ошибок компиляции. Также приводится описание реализации JIT-компилятора.

\section{Абстрактное синтаксическое дерево}
Выходными данными фронтенда компилятора является синтаксическое дерево следующего вида:
\begin{lstlisting}[language=Haskell]
data Type = 
    Number 
    | Reference Int 
    | Void

data Vardecl = Vardecl String Type

data Declaration = 
    VD Vardecl
    | Funcdecl Type String [Vardecl] Statement
    | Extdecl Type String [Vardecl] 
    
data Statement =
    Complex [Vardecl] [Statement]
    | Ite Expression Statement (Maybe Statement)
    | While Expression Statement
    | Expsta Expression
    | Return (Maybe Expression)
    
data Expression =
    ConstInt Int                      -- 7
    | ConstArr [Int]                  -- [7,8,9]
    | Takeval Expression              
    | Takeadr String                  
    | Call String [Expression]  
    | Assign [Expression] Expression  -- adr1 = adr2 = adr3 = 7
    | BracketOp Expression Expression
\end{lstlisting}
Абстракцией самого верхнего уровня является Declaration. Код C-- модуля де-факто представляет собой лес корневых элементов Declaration, отражающих инструкции объявления глобальных переменных и функций языка C--. 

\section{Трансляция}
\subsection{Переменные}
Переменные в АСТ представляются абстракцией Vardecl и имеют имя и тип. Глобальные переменные объявляются вне тел функций и представляются абстракцией VD.


Следующий пример LLVM IR объявляет глобальную переменную ``gvar'' типа ``i32'' с начальным значением 0 и глобальный i32 массив ``garr'' размера 10, инициализированный нулями.
\begin{lstlisting}[language=llvm]
@gvar = common global i32 0
@garr = common global <10 x i32> zeroinitializer
\end{lstlisting}
LLVM не предоставляет возможности объявить неинициализированную глобальную переменную, поэтому, несмотря на отсутствие поддержки инициализации переменных при объявлении в АСТ, компилятор производит инициализацию переменных нулевыми значениями, в зависимости от типа.

Ключевое слово ``common'' означает тип линковки переменной. Согласно документации, именно этот тип линковки используется для неинициализированных глобальных переменных в C-подобных языках. Далее так же будет использован тип ``external'' при объявлении внешних(библиотечных) функций, реализующих ввод-вывод.


Данная реализация компилятора поддерживает локальные переменные, однако производит выделение памяти на стеке а не в куче.


Пример объявления и инициализации локальной переменой:

\begin{lstlisting}[language=C]
//C-- some function scope:
int j;
j = 'Z';
\end{lstlisting}

\begin{lstlisting}[language=llvm]
; LLVM IR
%2 = alloca i32
store i32 90, i32* %2
\end{lstlisting}

\subsection{Функции}
Функции в АСТ представляются абстракцией Funcdecl и имеют: тип возвращаемого значения, имя, список параметров и тело.


Для удобства трансляции в АСТ была добавлена абстракция Extdecl, представляющая собой Funcdecl без тела, отражающая объявление внешних функций. Т.к. упрощенная версия языка C-- не поддерживает инструкций объявления внешних функций, Extdecl невозможно получить путем синтаксического разбора, т.е. добавление двух необходимых Extdecl(для реализации инструкций ``read'' и ``write'' языка C--) производится непосредтсвенно в коде компилятора, после этапа генерации АСТ.


Пример объявления внешней функции ``void putChar(int)'', отвечающей инструкции ``write'', и функции языка C--:
\begin{lstlisting}[language=C]
//C-- instruction ``write'', uses external putChar.
void print(int sym)
{
    write sym;
    return;
}
\end{lstlisting}

\begin{lstlisting}[language=llvm]
; LLVM IR
declare void @putChar(i32)

define void @print(i32 %sym) {
entry:
  %0 = alloca i32
  store i32 %sym, i32* %0
  %1 = load i32* %0
  call void @putChar(i32 %1)
  ret void
}
\end{lstlisting}

\subsection{Выражения}
АСТ поддерживает следующие типы выражений:
\begin{itemize}
  \item Целочисленная константа
  \item Целочисленный массив
  \item Вычисление значения выражения %ЗАЧЕМ?!
  \item Вычисление значения переменной
  \item Вызов функции
  \item Множественное присваивание (``a = b = 6'')
  \item Индексация в массиве
\end{itemize}
Ввиду особенности реализации фронтенда, по умолчанию АСТ поддерживает арифметические операции языка C-- посредством вызова ``Call <operator> [Expression]''. Т.к. LLVM поддерживает множество арифметических операций над различными типами данных, и язык C-- не является семантически объектно-ориентированным, т.е. не поддерживает перегрузку операторов, было принято решение транслировать Call арифметических операторов в существующие LLVM-инструкции вместо генерации кода поддержки операторов.


LLVM поддерживает объявление, в числе прочих, целочисленных констант и многомерных константных массивов. Вычисление значения переменной проихводится при помощи инструкции ``load'', вызов функции -- при помощи ``call''.


Пример сложного кода на C-- и его трансляции в LLVM IR:
\begin{lstlisting}[language=C]
//C--: m, k -- local, gvar -- global, print -- declared function
m = k = gvar = gvar + 1;
print(m);
\end{lstlisting}

\begin{lstlisting}[language=llvm]
; LLVM IR
%7 = load i32* @gvar
%8 = add i32 %7, 1
store i32 %8, i32* @gvar  ; gvar = gvar + 1
%9 = load i32* @gvar
store i32 %9, i32* %3     ; k = gvar
%10 = load i32* %3
store i32 %10, i32* %4    ; m = k
%11 = load i32* %4
call void @print(i32 %11) ; print(k)
\end{lstlisting}

\subsection{Ветвление}
АСТ поддерживает следующие инструкции ветвления:
\begin{itemize}
  \item while
  \item if-then
  \item if-then-else
\end{itemize}
Ветвление в LLVM производится при помощи создания нескольких блоков кода и объяления условных и безусловных переходов между ними.
Каждый блок, кроме первого и последнего должен иметь хотя бы один вход и один выход.

\subsubsection*{while}
\begin{lstlisting}[language=llvm]
; LLVM IR
while.cond:                                       ; preds = %if.exit, %entry
  ; ...
  br i1 %6, label %while.loop, label %while.exit

while.loop:                                       ; preds = %while.cond
  ; ...
  br label %while.cond

while.exit:                                       ; preds = %while.cond
  ; ...
\end{lstlisting}

\subsubsection*{if-then, if-then-else}
\begin{lstlisting}[language=llvm]
; LLVM IR
br i1 %13, label %if.then, label %if.else
if.then:                                          ; preds = %while.loop
  ; ...
  br label %if.exit

if.else:                                          ; preds = %while.loop
  ; ...
  br label %if.exit

if.exit:                                          ; preds = %if.else, %if.then
  ; ...
\end{lstlisting}
Инструкция ветвления с отсутствующим блоком ``else'' эмулирует поведение полной инструкции, т.е. создает блок ``if.else'' из единственной 
терминальной инструкции безусловного перехода в конец цикла. 

\subsection{Возврат управления}
В данной версии языка C-- инструкция возврата управления может присутствовать только в конце тела функции. В LLVM IR для возврата управления используется
инструкция ``ret''.

Примеры:
\begin{lstlisting}[language=llvm]
; function1
; ...
ret i32 %14
; function2
; ...
ret void
\end{lstlisting}

\section{Реализация}
Данный проект был реализован на языке Haskell с использованием библиотек llvm-general-pure и llvm-general, предоставляющих привязки для языка Haskell к C++ LLVM - библиотекам, поставляемым разработчиками LLVM и находящимися в открытом доступе.


Для реализации инструкций ``read'' и ``write'' была написана динамически линкуемая библиотека cbits.so.
\begin{lstlisting}[language=C]
#include <stdio.h>
#include <stdint.h>

void putChar(int32_t X) {
  putchar((char)X);
  fflush(stdout);
}

int32_t getChar() 
{
  return getchar();
}
\end{lstlisting}
Линковка и сборка Haskell-кода производится при помощи утилиты ``ghc'' (Glasgow Haskell Compiler, Version 7.10.1)


JIT-компиляция производится при помощи вызова LLVM ExecutionEngine, предоставляемого библиотекой llvm-general.
\end{document}